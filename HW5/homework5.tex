\documentclass{article}
\usepackage{graphicx} % Required for inserting images
\usepackage[utf8]{inputenc}
\usepackage{amsmath}
\usepackage{graphicx}
\usepackage{tikz}
\usepackage{array}
\usetikzlibrary{trees}
\usepackage{amssymb}
\usepackage{amsthm}
\usepackage{multirow}
\usepackage{dcolumn}
\usepackage{verbatim}

\newcolumntype{2}{D{.}{}{2.0}}

\title{CSC279 HW5}
\author{Hanzhang Yin}
\date{Nov/5/2023}

\begin{document}

\maketitle

\subsection*{Collaborator}
Chenxi Xu, Yekai Pan, Yiling Zou, Boyi Zhang

\section*{Question 15}

\subsection*{PROBLEM 15---Closest point, farthest point.}

\begin{enumerate}
    \item Assume $q$ is inside $P$. We want to find the closest point to $q$ in $\{p_1, \dots, p_n\}$.
    \item Assume $q$ is outside $P$. We want to find the closest point to $q$ in $\{p_1, \dots, p_n\}$.
    \item Assume $q$ is inside $P$. We want to find the farthest point to $q$ in $\{p_1, \dots, p_n\}$.
    \item Assume $q$ is outside $P$. We want to find the farthest point to $q$ in $\{p_1, \dots, p_n\}$.
    \item Assume $q$ is inside $P$. We want to find the closest point to $q$ on $P$.
    \item Assume $q$ is outside $P$. We want to find the closest point to $q$ on $P$.
    \item Assume $q$ is inside $P$. We want to find the farthest point to $q$ on $P$.
    \item Assume $q$ is outside $P$. We want to find the farthest point to $q$ on $P$.
\end{enumerate}
\\
\textbf{Answer: }
\\
\textbf{Question 1 - 4 Reasoning: }
The distance from $q$ to the vertices $p_i$ is unimodal function along the ordered sequence. 
This is true whether $q$ is inside or outside $P$. Along the vertices, we can use ternary (binary) search on sequence of vertices to find MIN or MAX distance.

\begin{enumerate}
    \item \textbf{Problem 1:} \( q \) inside \( P \); find the closest point to \( q \) in \( \{p_1, \dots, p_n\} \).
    \begin{itemize}
        \item \textbf{Solution:} \( O(\log n) \) time.
        \item \textbf{Reasoning:} The distance function is unimodal along the vertices, allowing ternary search.
    \end{itemize}

    \item \textbf{Problem 2:} \( q \) outside \( P \); find the closest point to \( q \) in \( \{p_1, \dots, p_n\} \).
    \begin{itemize}
        \item \textbf{Solution:} \( O(\log n) \) time.
        \item \textbf{Reasoning:} Similar to Problem 1, use ternary search due to the unimodal distance function.
    \end{itemize}

    \item \textbf{Problem 3:} \( q \) inside \( P \); find the farthest point from \( q \) in \( \{p_1, \dots, p_n\} \).
    \begin{itemize}
        \item \textbf{Solution:} \( O(\log n) \) time.
        \item \textbf{Reasoning:} Unimodal distance function allows ternary search for the maximum.
    \end{itemize}

    \item \textbf{Problem 4:} \( q \) outside \( P \); find the farthest point from \( q \) in \( \{p_1, \dots, p_n\} \).
    \begin{itemize}
        \item \textbf{Solution:} \( O(\log n) \) time.
        \item \textbf{Reasoning:} Same as Problem 3, apply ternary search on the unimodal distance function.
    \end{itemize}
\end{enumerate}
\\
\textbf{Question 5 - 6 Reasoning: }
\\
The distance from $q$ to the boundary of the convex polygon $P$ is a convex function along the perimeter, regardless of whether $q$ is inside or outside $P$.
(i.e. The distance decreases to a minimum point and then increases, forming a single through)

\begin{enumerate}
    \item \textbf{Problem 5:} \( q \) inside \( P \); find the closest point to \( q \) on \( P \).
    \begin{itemize}
        \item \textbf{Solution:} \( O(\log n) \) time.
        \item \textbf{Reasoning:} Perform binary search over edges to find the closest point where the minimum distance occurs.
    \end{itemize}

    \item \textbf{Problem 6:} \( q \) outside \( P \); find the closest point to \( q \) on \( P \).
    \begin{itemize}
        \item \textbf{Solution:} \( O(\log n) \) time.
        \item \textbf{Reasoning:} Same as Problem 5, use binary search to find the closest point on edges.
    \end{itemize}
\end{enumerate}
\\
\textbf{Question 7 - 8 Reasoning: }
\\
The farthest point from $q$ can lie anywhere along the perimeter of the convex polygon $P$, necessitating a check of all edges and vertices.
The distance function for farthest points is not unimodal, preventing the use of efficient binary or ternary search methods.
Hence, without preprocessing, we need $\Omega(n)$ time to do them.

\begin{enumerate}
    \item \textbf{Problem 7:} \( q \) inside \( P \); find the farthest point from \( q \) on \( P \).
    \begin{itemize}
        \item \textbf{Solution:} \( \Omega(n) \) time.
        \item \textbf{Reasoning:} Requires examining all edges using rotating calipers for antipodal points.
    \end{itemize}

    \item \textbf{Problem 8:} \( q \) outside \( P \); find the farthest point from \( q \) on \( P \).
    \begin{itemize}
        \item \textbf{Solution:} \( \Omega(n) \) time.
        \item \textbf{Reasoning:} Similar to Problem 7, needs \( \Omega(n) \) time to check all edges.
    \end{itemize}
\end{enumerate}


\subsection*{PROBLEM 16}

\begin{proof}
    For a n-polygon, 

\end{proof}

\subsection*{PROBLEM 17}

\begin{verbatim}
import random
import matplotlib.pyplot as plt
from collections import deque

Input:
    - P = {p1, p2, ..., pn}: Set of n points in the plane
    - r: Radius of each circle Ci
    - l: Radius of the target circle Q, where l > r

Output:
    - G: Set of "good" points

Algorithm FindGoodPoints(p_1, ..., p_n, r, l):

1. Build the Delaunay triangulation (DT) of the points {p_1, ..., p_n}.
    - Time complexity: O(n log n)
    
2. Initialize an empty list GoodPoints.
    
3. For each point p_i in {p_1, ..., p_n}:
    - Time complexity: O(n)
    a. Initialize an empty list of intervals I_i.
    
    b. For each neighbor p_j of p_i in DT:
        - Time complexity: O(1).
        i. Compute d = distance between p_i and p_j.
    
        ii. If d <= 2l:
            - Compute theta_j = angle between vector (p_j - p_i) and the x-axis.
            - Compute phi_j = arccos(d / (2 * (l + r))).
                (Ensure the argument of arccos is within [-1, 1].)
    
            - If phi_j is real:
                - Add the interval [theta_j - phi_j, theta_j + phi_j] to I_i.
    
    c. Sort the intervals in I_i.
        - Time complexity: O(1), every Voronoi point have constant neighbor.
        - Merge overlapping intervals to get the union.
        
    
    d. If the union of intervals in I_i covers [0, 2pi]:
        - p_i is not good.
    e. Else:
        - p_i is good.
        - Add p_i to GoodPoints.
    
4. Return GoodPoints.
\end{verbatim}

\subsection*{PROBLEM 18}
The following code of randomized Delaunay triangulation algorithm with history DAG was implemented in Python:
\begin{verbatim}
# ----------------------------
# Data Structures Definitions
# ----------------------------

class Point:
    def __init__(self, x, y):
        self.x = x
        self.y = y

class SegmentWrapper:
    """
    Helper class to uniquely identify a segment irrespective of point order.
    """
    def __init__(self, p1, p2):
        # Ensure consistent ordering
        if (p1.x, p1.y) < (p2.x, p2.y):
            self.p1, self.p2 = p1, p2
        else:
            self.p1, self.p2 = p2, p1

    def __eq__(self, other):
        return (self.p1.x, self.p1.y) == (other.p1.x, other.p1.y) and \
               (self.p2.x, self.p2.y) == (other.p2.x, other.p2.y)

    def __hash__(self):
        return hash(((self.p1.x, self.p1.y), (self.p2.x, self.p2.y)))

    def __repr__(self):
        return f"Segment(({self.p1.x}, {self.p1.y}) - ({self.p2.x}, {self.p2.y}))"

class Triangle:
    def __init__(self, p1, p2, p3):
        self.vertices = [p1, p2, p3]  # Points
        self.edges = [
            SegmentWrapper(p1, p2),
            SegmentWrapper(p2, p3),
            SegmentWrapper(p3, p1)
        ]
        self.neighbors = {}  # edge -> adjacent triangle
        self.children = []  # For history DAG

    def contains_point(self, point):
        return point in self.vertices

    def __repr__(self):
        verts = ', '.join([f"({p.x}, {p.y})" for p in self.vertices])
        return f"Triangle({verts})"

class HistoryDAGNode:
    def __init__(self, triangle):
        self.triangle = triangle  # Triangle object
        self.children = []        # List of HistoryDAGNode

    def add_child(self, child_node):
        self.children.append(child_node)

# ----------------------------
# Delaunay Triangulation Class
# ----------------------------

class DelaunayTriangulation:
    def __init__(self, points):
        self.points = points  # List of Point objects
        self.segments = {}    # SegmentWrapper -> list of adjacent Triangles
        self.triangles = []   # List of Triangle objects
        self.history_root = None

    def create_super_triangle(self):
        """
        Create a super-triangle that encompasses all the points.
        """
        min_x = min(p.x for p in self.points)
        max_x = max(p.x for p in self.points)
        min_y = min(p.y for p in self.points)
        max_y = max(p.y for p in self.points)

        dx = max_x - min_x
        dy = max_y - min_y
        delta_max = max(dx, dy) * 100  # Make it large enough

        # Create three points that form a super-triangle
        p1 = Point(min_x - delta_max, min_y - delta_max)
        p2 = Point(min_x + 2 * delta_max, min_y - delta_max)
        p3 = Point(min_x - delta_max, max_y + 2 * delta_max)

        super_triangle = Triangle(p1, p2, p3)
        self.triangles.append(super_triangle)
        self.history_root = HistoryDAGNode(super_triangle)

        # Add segments of the super-triangle
        for edge in super_triangle.edges:
            self.segments.setdefault(edge, []).append(super_triangle)

    def locate_containing_triangle(self, point):
        """
        Traverse the history DAG to locate the triangle containing the point.
        """
        node = self.history_root
        while node.children:
            found = False
            for child in node.children:
                if self.point_in_triangle(point, child.triangle):
                    node = child
                    found = True
                    break
            if not found:
                break  # Point not found in any child; fallback
        # Now, node.triangle should contain the point
        return node.triangle, node

    @staticmethod
    def point_in_triangle(p, triangle):
        """
        Check if point p is inside the given triangle using barycentric coordinates.
        """
        def sign(p1, p2, p3):
            return (p1.x - p3.x) * (p2.y - p3.y) - \
                   (p2.x - p3.x) * (p1.y - p3.y)

        b1 = sign(p, triangle.vertices[0], triangle.vertices[1]) < 0.0
        b2 = sign(p, triangle.vertices[1], triangle.vertices[2]) < 0.0
        b3 = sign(p, triangle.vertices[2], triangle.vertices[0]) < 0.0

        return ((b1 == b2) and (b2 == b3))

    def insert_point(self, point):
        """
        Insert a single point into the triangulation.
        """
        containing_triangle, containing_node = self.locate_containing_triangle(point)

        # Remove the containing triangle
        self.triangles.remove(containing_triangle)

        # Create new triangles by connecting the point to the vertices of the containing triangle
        t1 = Triangle(point, containing_triangle.vertices[0], containing_triangle.vertices[1])
        t2 = Triangle(point, containing_triangle.vertices[1], containing_triangle.vertices[2])
        t3 = Triangle(point, containing_triangle.vertices[2], containing_triangle.vertices[0])

        # Set neighbors
        self.set_neighbors(t1, containing_triangle)
        self.set_neighbors(t2, containing_triangle)
        self.set_neighbors(t3, containing_triangle)

        # Add new triangles
        self.triangles.extend([t1, t2, t3])

        # Update segments
        for t in [t1, t2, t3]:
            for edge in t.edges:
                self.segments.setdefault(edge, []).append(t)

        # Update history DAG
        child_nodes = [
            HistoryDAGNode(t1),
            HistoryDAGNode(t2),
            HistoryDAGNode(t3)
        ]
        for child in child_nodes:
            containing_node.add_child(child)

        # Edge Legalization
        for t in [t1, t2, t3]:
            for edge in t.edges:
                if point in [edge.p1, edge.p2]:
                    self.legalize_edge(edge, t, point)

    def set_neighbors(self, new_triangle, old_triangle):
        """
        Set the neighboring triangles for the new triangle.
        """
        for edge in new_triangle.edges:
            if edge in self.segments:
                for neighbor in self.segments[edge]:
                    if neighbor != old_triangle and neighbor != new_triangle:
                        new_triangle.neighbors[edge] = neighbor
                        neighbor.neighbors[edge] = new_triangle

    def legalize_edge(self, edge, triangle, point):
        """
        Legalize an edge to restore the Delaunay condition.
        """
        if edge not in triangle.neighbors:
            return  # Boundary edge

        neighbor = triangle.neighbors[edge]
        if neighbor is None:
            return

        # Find the opposite point in the neighbor triangle
        opposite_point = [v for v in neighbor.vertices if v not in [edge.p1, edge.p2]][0]

        if self.in_circumcircle(opposite_point, triangle):
            # Perform edge flip
            new_triangles = self.edge_flip(triangle, neighbor, edge, point, opposite_point)
            for new_t in new_triangles:
                for e in new_t.edges:
                    if point in [e.p1, e.p2]:
                        self.legalize_edge(e, new_t, point)

    def edge_flip(self, t1, t2, edge, point, opposite_point):
        """
        Flip the shared edge between two triangles.
        """
        # Remove old triangles
        self.triangles.remove(t1)
        self.triangles.remove(t2)

        # Remove edge from segments
        self.segments[edge].remove(t1)
        self.segments[edge].remove(t2)
        if not self.segments[edge]:
            del self.segments[edge]

        # Create new edge
        new_edge = SegmentWrapper(point, opposite_point)

        # Create new triangles
        new_t1 = Triangle(point, edge.p1, opposite_point)
        new_t2 = Triangle(point, opposite_point, edge.p2)

        # Update segments
        for t in [new_t1, new_t2]:
            for e in t.edges:
                self.segments.setdefault(e, []).append(t)

        # Update neighbors
        self.update_neighbors_after_flip(t1, t2, new_t1, new_t2, edge, new_edge)

        # Add new triangles
        self.triangles.extend([new_t1, new_t2])

        # Update history DAG
        parent_node = HistoryDAGNode(None)
        t1_node = self.find_history_node(self.history_root, t1)
        t2_node = self.find_history_node(self.history_root, t2)
        parent_node.add_child(t1_node)
        parent_node.add_child(t2_node)
        new_t1_node = HistoryDAGNode(new_t1)
        new_t2_node = HistoryDAGNode(new_t2)
        parent_node.add_child(new_t1_node)
        parent_node.add_child(new_t2_node)

        return [new_t1, new_t2]

    def update_neighbors_after_flip(self, t1, t2, new_t1, new_t2, old_edge, new_edge):
        """
        Update neighbor relationships after an edge flip.
        """
        # Set neighbors for new_t1
        new_t1.neighbors[new_edge] = new_t2
        new_t2.neighbors[new_edge] = new_t1

        # Update other neighbors
        for e in new_t1.edges:
            if e != new_edge:
                for neighbor in self.segments[e]:
                    if neighbor != new_t1:
                        new_t1.neighbors[e] = neighbor
                        neighbor.neighbors[e] = new_t1

        for e in new_t2.edges:
            if e != new_edge:
                for neighbor in self.segments[e]:
                    if neighbor != new_t2:
                        new_t2.neighbors[e] = neighbor
                        neighbor.neighbors[e] = new_t2

    def find_history_node(self, node, triangle):
        """
        Find the history DAG node corresponding to the given triangle.
        """
        if node.triangle == triangle:
            return node
        for child in node.children:
            result = self.find_history_node(child, triangle)
            if result is not None:
                return result
        return None

    def in_circumcircle(self, point, triangle):
        """
        Check if a point is inside the circumcircle of a triangle.
        """
        ax, ay = triangle.vertices[0].x - point.x, triangle.vertices[0].y - point.y
        bx, by = triangle.vertices[1].x - point.x, triangle.vertices[1].y - point.y
        cx, cy = triangle.vertices[2].x - point.x, triangle.vertices[2].y - point.y

        det = (ax * (by * (cx**2 + cy**2) - cy * (bx**2 + by**2)) -
               ay * (bx * (cx**2 + cy**2) - cx * (bx**2 + by**2)) +
               (ax**2 + ay**2) * (bx * cy - cx * by))

        return det > 0

    def build_triangulation(self):
        """
        Build the triangulation by inserting all points.
        """
        self.create_super_triangle()
        for point in self.points:
            self.insert_point(point)
        self.remove_super_triangle()

    def remove_super_triangle(self):
        """
        Remove any triangles that share a vertex with the super-triangle.
        """
        # Super-triangle vertices
        super_vertices = set(self.history_root.triangle.vertices)
        self.triangles = [t for t in self.triangles if not any(v in super_vertices for v in t.vertices)]

    def calculate_dag_depths(self):
        """
        Calculate maximum and average depths of the history DAG.
        """
        depths = []
        queue = deque([(self.history_root, 0)])

        while queue:
            node, depth = queue.popleft()
            if not node.children:
                depths.append(depth)
            else:
                for child in node.children:
                    queue.append((child, depth + 1))

        max_depth = max(depths) if depths else 0
        avg_depth = sum(depths) / len(depths) if depths else 0
        return max_depth, avg_depth
    
    def plot_triangulation(self):
        """
        Plot the triangulation.
        """
        plt.figure(figsize=(8, 8))
        for triangle in self.triangles:
            x_coords = [vertex.x for vertex in triangle.vertices + [triangle.vertices[0]]]
            y_coords = [vertex.y for vertex in triangle.vertices + [triangle.vertices[0]]]
            plt.plot(x_coords, y_coords, 'k-')
        plt.scatter([p.x for p in self.points], [p.y for p in self.points], color='red', s=10)
        plt.axis('equal')
        plt.title('Delaunay Triangulation')
        plt.show()
\end{verbatim}


\end{document}
