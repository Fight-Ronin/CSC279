\documentclass{article}
\usepackage{graphicx} % Required for inserting images
\usepackage[utf8]{inputenc}
\usepackage{amsmath}
\usepackage{graphicx}
\usepackage{tikz}
\usepackage{array}
\usetikzlibrary{trees}
\usepackage{amssymb}
\usepackage{amsthm}
\usepackage{multirow}
\usepackage{dcolumn}
\usepackage{verbatim}

\newcolumntype{2}{D{.}{}{2.0}}

\title{CSC279 HW5}
\author{Hanzhang Yin}
\date{Nov/5/2023}

\begin{document}

\maketitle

\subsection*{Collaborator}
Chenxi Xu, Yekai Pan, Yiling Zou, Boyi Zhang

\section*{Question 15}

\subsection*{PROBLEM 15---Closest point, farthest point.}

\begin{enumerate}
    \item Assume $q$ is inside $P$. We want to find the closest point to $q$ in $\{p_1, \dots, p_n\}$.
    \item Assume $q$ is outside $P$. We want to find the closest point to $q$ in $\{p_1, \dots, p_n\}$.
    \item Assume $q$ is inside $P$. We want to find the farthest point to $q$ in $\{p_1, \dots, p_n\}$.
    \item Assume $q$ is outside $P$. We want to find the farthest point to $q$ in $\{p_1, \dots, p_n\}$.
    \item Assume $q$ is inside $P$. We want to find the closest point to $q$ on $P$.
    \item Assume $q$ is outside $P$. We want to find the closest point to $q$ on $P$.
    \item Assume $q$ is inside $P$. We want to find the farthest point to $q$ on $P$.
    \item Assume $q$ is outside $P$. We want to find the farthest point to $q$ on $P$.
\end{enumerate}
\\
\textbf{Answer: }
\\
\textbf{Question 1 - 4 Reasoning: }
The distance from $q$ to the vertices $p_i$ is unimodal function along the ordered sequence. 
This is true whether $q$ is inside or outside $P$. Along the vertices, we can use ternary (binary) search on sequence of vertices to find MIN or MAX distance.

\begin{enumerate}
    \item \textbf{Problem 1:} \( q \) inside \( P \); find the closest point to \( q \) in \( \{p_1, \dots, p_n\} \).
    \begin{itemize}
        \item \textbf{Solution:} \( O(\log n) \) time.
        \item \textbf{Reasoning:} The distance function is unimodal along the vertices, allowing ternary search.
    \end{itemize}

    \item \textbf{Problem 2:} \( q \) outside \( P \); find the closest point to \( q \) in \( \{p_1, \dots, p_n\} \).
    \begin{itemize}
        \item \textbf{Solution:} \( O(\log n) \) time.
        \item \textbf{Reasoning:} Similar to Problem 1, use ternary search due to the unimodal distance function.
    \end{itemize}

    \item \textbf{Problem 3:} \( q \) inside \( P \); find the farthest point from \( q \) in \( \{p_1, \dots, p_n\} \).
    \begin{itemize}
        \item \textbf{Solution:} \( O(\log n) \) time.
        \item \textbf{Reasoning:} Unimodal distance function allows ternary search for the maximum.
    \end{itemize}

    \item \textbf{Problem 4:} \( q \) outside \( P \); find the farthest point from \( q \) in \( \{p_1, \dots, p_n\} \).
    \begin{itemize}
        \item \textbf{Solution:} \( O(\log n) \) time.
        \item \textbf{Reasoning:} Same as Problem 3, apply ternary search on the unimodal distance function.
    \end{itemize}
\end{enumerate}
\\
\textbf{Question 5 - 6 Reasoning: }
\\
The distance from $q$ to the boundary of the convex polygon $P$ is a convex function along the perimeter, regardless of whether $q$ is inside or outside $P$.
(i.e. The distance decreases to a minimum point and then increases, forming a single through)

\begin{enumerate}
    \item \textbf{Problem 5:} \( q \) inside \( P \); find the closest point to \( q \) on \( P \).
    \begin{itemize}
        \item \textbf{Solution:} \( O(\log n) \) time.
        \item \textbf{Reasoning:} Perform binary search over edges to find the closest point where the minimum distance occurs.
    \end{itemize}

    \item \textbf{Problem 6:} \( q \) outside \( P \); find the closest point to \( q \) on \( P \).
    \begin{itemize}
        \item \textbf{Solution:} \( O(\log n) \) time.
        \item \textbf{Reasoning:} Same as Problem 5, use binary search to find the closest point on edges.
    \end{itemize}
\end{enumerate}
\\
\textbf{Question 7 - 8 Reasoning: }
\\
The farthest point from $q$ can lie anywhere along the perimeter of the convex polygon $P$, necessitating a check of all edges and vertices.
The distance function for farthest points is not unimodal, preventing the use of efficient binary or ternary search methods.
Hence, without preprocessing, we need $\Omega(n)$ time to do them.

\begin{enumerate}
    \item \textbf{Problem 7:} \( q \) inside \( P \); find the farthest point from \( q \) on \( P \).
    \begin{itemize}
        \item \textbf{Solution:} \( \Omega(n) \) time.
        \item \textbf{Reasoning:} Requires examining all edges using rotating calipers for antipodal points.
    \end{itemize}

    \item \textbf{Problem 8:} \( q \) outside \( P \); find the farthest point from \( q \) on \( P \).
    \begin{itemize}
        \item \textbf{Solution:} \( \Omega(n) \) time.
        \item \textbf{Reasoning:} Similar to Problem 7, needs \( \Omega(n) \) time to check all edges.
    \end{itemize}
\end{enumerate}


\subsection*{PROBLEM 16}

\begin{proof}


\end{proof}

\subsection*{PROBLEM 17}

\begin{verbatim}
Input:
    - P = {p1, p2, ..., pn}: Set of n points in the plane
    - r: Radius of each circle Ci
    - l: Radius of the target circle Q, where l > r

Output:
    - G: Set of "good" points

Algorithm FindGoodPoints(p_1, ..., p_n, r, l):

1. Build the Delaunay triangulation (DT) of the points {p_1, ..., p_n}.
    - Time complexity: O(n log n)
    
2. Initialize an empty list GoodPoints.
    
3. For each point p_i in {p_1, ..., p_n}:
    - Time complexity: O(n)
    a. Initialize an empty list of intervals I_i.
    
    b. For each neighbor p_j of p_i in DT:
        - Time complexity: O(1).
        i. Compute d = distance between p_i and p_j.
    
        ii. If d <= 2l:
            - Compute theta_j = angle between vector (p_j - p_i) and the x-axis.
            - Compute phi_j = arccos(d / (2 * (l + r))).
                (Ensure the argument of arccos is within [-1, 1].)
    
            - If phi_j is real:
                - Add the interval [theta_j - phi_j, theta_j + phi_j] to I_i.
    
    c. Sort the intervals in I_i.
        - Time complexity: O(1), every Voronoi point have constant neighbor.
        - Merge overlapping intervals to get the union.
        
    
    d. If the union of intervals in I_i covers [0, 2pi]:
        - p_i is not good.
    e. Else:
        - p_i is good.
        - Add p_i to GoodPoints.
    
4. Return GoodPoints.
\end{verbatim}

\subsection*{PROBLEM 18}



\end{document}
