\documentclass{article}
\usepackage{graphicx} % Required for inserting images
\usepackage[utf8]{inputenc}
\usepackage{amsmath}
\usepackage{graphicx}
\usepackage{tikz}
\usepackage{array}
\usetikzlibrary{trees}
\usepackage{amssymb}
\usepackage{amsthm}
\usepackage{multirow} 
\usepackage{dcolumn}
\usepackage{verbatim}

\newcolumntype{2}{D{.}{}{2.0}}

\title{CSC279 HW4 - Open Problem Exploration}
\author{Hanzhang Yin}
\date{Oct/3/2023}

\begin{document}

\maketitle

Picked Probelm: \textbf{Simple Linear-Time Polygon Triangulation} 
\\
\url{https://topp.openproblem.net/p10}

\subsection*{Problem Description:}
Polygon triangulation involves partitioning a simple polygon into non-overlapping triangles.
The primary challenge lies in achieving efficient algorithms that perform this triangulation in linear time, especially for large and complex polygons.

\subsection*{Approaches and Recent Developments:}
Chazelle's [1] groundbreaking 1991 algorithm was the first to achieve deterministic linear-time triangulation of a simple polygon. However, its complexity has driven researchers to seek simpler yet efficient alternatives. Recent (But not that recent) advancements include both deterministic and randomized algorithms that strive to match or approach linear-time performance with greater simplicity:
\begin{enumerate}
    \item \textbf{Deterministic Linear-Time Algorithm: }
    A new approach leverages the polygon-cutting and planar separator theorems to build a coarse triangulation in a bottom-up phase, followed by a top-down refinement [2]. This method avoids complex data structures like dynamic search trees, relying instead on elementary structures, thus simplifying implementation compared to Chazelle's algorithm.
    \item \textbf{Randomized Algorithms: }
    One algorithm computes the trapezoidal decomposition of a simple polygon in expected linear time, enabling linear-time triangulation through known reductions. It simplifies Chazelle's method by performing random sampling  [3] on subchains of the polygon rather than its edges.
\end{enumerate}

\subsection*{Possible Thoughts For Approaching (Topological Approach):}
To develop a simple linear-time algorithm for polygon triangulation, we can use the ear clipping method. [4]
First, decompose the polygon into monotone pieces using an \( O(n) \) algorithm (there exists a few currently).
Then, apply a simplified ear clipping method to each monotone piece.
An "ear" is a triangle formed by three consecutive vertices that lies entirely inside the polygon without containing any other vertices. 
Since ear clipping operates efficiently on monotone polygons, this approach achieves an overall linear-time (\( O(n) + O(n) \sim O(2n) \)) algorithm and might be considered running in a simpler manner by avoiding complex data structures. 
By clipping off ears sequentially, the polygon is completely and efficiently triangulated.

\subsection*{Further Exploration: }

\begin{thebibliography}{3}
    \bibitem{Cha91}
    Chazelle, Bernard. 1991. “Triangulating a Simple Polygon in Linear Time.” \textit{Discrete \& Computational Geometry}, 6(3): 485-524. https://doi.org/10.1007/BF02574703.

    \bibitem{Amato2000}
    Amato, Nancy M., Michael T. Goodrich, and Edgar A. Ramos. 2000. “Linear-Time Triangulation of a Simple Polygon Made Easier via Randomization.” In \textit{Proceedings of the 16th Annual Symposium on Computational Geometry}, 201–212. https://doi.org/10.1145/336154.336206.

    \bibitem{Seidel1991}
    Seidel, Raimund. 1991. “A Simple and Fast Incremental Randomized Algorithm for Computing Trapezoidal Decompositions and for Triangulating Polygons.” \textit{Computational Geometry}, 1(1): 51–64. https://doi.org/10.1016/0925-7721(91)90012-4.

    \bibitem{Garey1978}
    Garey, M. R., Johnson, D. S., Preparata, F. P., & Tarjan, R. E. (1978). Triangulating a Simple Polygon. Information Processing Letters, 7(4), 175-179
\end{thebibliography}


\end{document}
